\subsection*{题目背景}

期末要到了,小菜同学找到了自己的好朋友顿顿,希望可以预测一下自己这学期是否会挂科。


\subsection*{题目描述}

首先,顿顿选取了如“课堂表现”、“自习时长”、“作业得分”、“社团活动参与度”等 $n$ 项指标作为预测的依据。
然后,顿顿根据自己平日里对小菜的暗中观察,以百分制给每一项指标打分,即小菜同学第 $i$($1 \le i \le n$)项指标的得分 $score_i$ 是一个 $\left[ 0, 100 \right]$ 范围内的整数。
鉴于每一项指标的重要性不尽相同,顿顿用一个 $\left[ -10, 10 \right]$ 范围内的整数 $w_i$ 来表示第 $i$($1 \le i \le n$)项指标的重要程度。

最后,小菜同学期末的安全指数 $y$ 定义如下:

\begin{equation*}
    y = \mathrm{ReLU} \left( \sum\limits_{i=1}^{n} score_i \cdot w_i \right)
\end{equation*}


其中 $\mathrm{ReLU} \left( x \right) = \max \left( 0, x \right)$ 是一种常见的激活函数。
因为使用了 $\mathrm{ReLU}$ 函数,安全指数一定是个非负值。
如果安全指数过低(甚至为零),则说明小菜同学这学期很可能要挂科了……

已知每一项指标的重要程度 $w_i$ 和相应的得分 $score_i$,快来算算小菜同学期末的安全指数吧。


\subsection*{输入格式}

从标准输入读入数据。

输入的第一行包含一个正整数 $n$,保证 $2 \le n \le 10^{5}$。

接下来输入 $n$ 行,其中第 $i$($1 \le i \le n$)行包含用空格分隔的两个整数 $w_i$ 和 $score_i$,分别表示第 $i$ 项指标的重要程度和小菜同学该项的得分。


\subsection*{输出格式}

输出到标准输出。

输出一个非负整数 $y$,表示小菜同学期末的安全指数。

\examplebox*{\lstinputlisting[frame=none]{data/21/1-1.in}}{\lstinputlisting[frame=none]{data/21/1-1.out}}

$y = \mathrm{ReLU}(1220)=1220$

\examplebox*{\lstinputlisting[frame=none]{data/21/1-2.in}}{\lstinputlisting[frame=none]{data/21/1-2.out}}

$y = \mathrm{ReLU}(-1015)=0$

\subsection*{子任务}

对于 $100\%$ 的数据,满足 $2\le n\le 10^5, 0\le score_i\le 100, -10\le w_i\le 10$。