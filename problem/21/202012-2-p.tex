\subsection*{题目背景}

考虑到安全指数是一个较大范围内的整数、小菜很可能搞不清楚自己是否真的安全,顿顿决定设置一个阈值 $\theta$,以便将安全指数 $y$ 转化为一个具体的预测结果——“会挂科”或“不会挂科”。

因为安全指数越高表明小菜同学挂科的可能性越低,所以当 $y \ge \theta$ 时,顿顿会预测小菜这学期很安全、不会挂科;反之若 $y < \theta$,顿顿就会劝诫小菜:“你期末要挂科了,勿谓言之不预也。”

那么这个阈值该如何设定呢?顿顿准备从过往中寻找答案。


\subsection*{题目描述}

具体来说,顿顿评估了 $m$ 位同学上学期的安全指数,其中第 $i$($1 \le i \le m$)位同学的安全指数为 $y_i$,是一个 $[ 0, 10^8 ]$ 范围内的整数;同时,该同学上学期的挂科情况记作 $result_i \in { 0, 1 }$,其中 $0$ 表示挂科、$1$ 表示未挂科。

相应地,顿顿用 $predict_{\theta} ( y )$ 表示根据阈值 $\theta$ 将安全指数 $y$ 转化为的具体预测结果。
如果 $predict_{\theta} ( y_j )$ 与 $result_j$ 相同,则说明阈值为 $\theta$ 时顿顿对第 $j$ 位同学是否挂科预测正确;不同则说明预测错误。

\begin{equation*}
    \mathrm{predict}_{\theta} ( y ) =
    \begin{cases}
        0 & (y < \theta)   \\
        1 & (y \ge \theta) \\
    \end{cases}
\end{equation*}


最后,顿顿设计了如下公式来计算最佳阈值 $\theta^*$:

\begin{equation*}
    \theta^* = \max { \mathop{\mathrm{argmax} }\limits_{\theta \in { y_i } } \sum\limits_{j=1}^{m} ( \mathrm{predict}_{\theta} ( y_j ) == result_j ) }
\end{equation*}


该公式亦可等价地表述为如下规则:

\begin{enumerate}

    \item 最佳阈值仅在 ${ y_i }$ 中选取,即与某位同学的安全指数相同;
    \item 按照该阈值对这 $m$ 位同学上学期的挂科情况进行预测,预测正确的次数最多(即准确率最高);



    \item 多个阈值均可以达到最高准确率时,选取其中最大的。



\end{enumerate}


\subsection*{输入格式}

从标准输入读入数据。

输入的第一行包含一个正整数 $m$。

接下来输入 $m$ 行,其中第 $i$($1 \le i \le m$)行包括用空格分隔的两个整数 $y_i$ 和 $result_i$,含义如上文所述。


\subsection*{输出格式}

输出到标准输出。

输出一个整数,表示最佳阈值 $\theta^*$。

\examplebox*{\lstinputlisting[frame=none]{data/21/2-1.in}}{\lstinputlisting[frame=none]{data/21/2-1.out}}

按照规则一,最佳阈值的选取范围为 $0, 1, 3, 5, 7$。

$\theta = 0$ 时,预测正确次数为 $4$;

$\theta = 1$ 时,预测正确次数为 $5$;

$\theta = 3$ 时,预测正确次数为 $5$;

$\theta = 5$ 时,预测正确次数为 $4$;

$\theta = 7$ 时,预测正确次数为 $3$。

阈值选取为 $1$ 或 $3$ 时,预测准确率最高;
所以按照规则二,最佳阈值的选取范围缩小为 $1,3$。

依规则三,$\theta^*=\max\{1,3\}=3$
。


\subsection*{子任务}

$70\%$ 的测试数据保证 $m \le 200$;

全部的测试数据保证 $2 \le m \le 10^{5}$。


