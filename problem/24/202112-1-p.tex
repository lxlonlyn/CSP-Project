\subsection*{题目背景}

西西艾弗岛的购物中心里店铺林立,商品琳琅满目。为了帮助游客根据自己的预算快速选择心仪的商品,IT 部门决定研发一套商品检索系统,支持对任意给定的预算 $x$,查询在该预算范围内($\le x$)价格最高的商品。如果没有商品符合该预算要求,便向游客推荐可以免费领取的西西艾弗岛定制纪念品。

假设购物中心里有 $n$ 件商品,价格从低到高依次为 $A_1,A_2,\cdots, A_n$,
,则根据预算 $x$ 检索商品的过程可以抽象为如下序列查询问题。

\subsection*{题目描述}

$A=[A_0,A_1,A_2,\cdots,A_n]$ 是一个由 $n+1$ 个 $[0,N)$ 范围内整数组成的序列,满足 $0=A_0<A_1<A_2<\cdots<A_n<N$
。(这个定义中蕴含了 $n$ 一定小于 $N$。)

基于序列 $A$,对于 $[0,N)$ 范围内任意的整数 $x$,查询 $f(x)$ 定义为:序列 $A$ 中{\heiti 小于等于} $x$ 的整数里最大的数的{\heiti 下标}。具体来说有以下两种情况:

\begin{enumerate}
  \item 存在下标 $0\le i<n$ 满足 $A_i\le x < A_{i+1}$,此时序列 $A$ 中从 $A_0$ 到 $A_i$ 均小于等于 $x$,其中最大的数为 $A_i$,其下标为 $i$,故 $f(x)=i$。
  \item $A_n\le x$,此时序列 $A$ 中左右的数都小于等于 $x$,其中最大的数是 $A_n$,故 $f(x)=n$。
\end{enumerate}

令 $sum(A)$ 表示 $f(0)$ 到 $f(N-1)$ 的总和,即:

\begin{equation*}
sum(A) = \sum_{i=0}^{N-1} {f(i)=f(0)+f(1)+f(2)+\cdots +f(N-1)}
\end{equation*}

对于给定的序列 $A$,试计算 $sum(A)$。

\subsection*{输入格式}

从标准输入读入数据。

输入的第一行包含空格分隔的两个正整数 $n$ 和 $N$。

输入的第二行包含 $n$ 个用空格分隔的整数 $A_1,A_2,\cdots,A_n$
。

注意 $A_0$
 固定为 $0$,因此输入数据中不包括 $A_0$ 
。

\subsection*{输出格式}

输出到标准输出。

仅输出一个整数,表示 $sum(A)$ 的值。

\examplebox*{\lstinputlisting[frame=none]{data/24/1-1.in}}{\lstinputlisting[frame=none]{data/24/1-1.out}}

$A=[0, 2, 5, 8]$

\begin{table}[H]
  \centering
  \begin{tabular}{ccccccccccc}
    \toprule
    $i$ & $0$ & $1$ & $2$ & $3$ & $4$ & $5$ & $6$ & $7$ & $8$ & $9$ \\
    $f(i)$ & $0$ & $0$ & $1$ & $1$ & $1$ & $2$ & $2$ & $2$ & $3$ & $3$ \\
    \bottomrule
  \end{tabular}
\end{table}

如上表所示,$sum(A)=f(0)+f(1)+\cdots + f(9)=15$。

考虑到 $f(0)=f(1)$、$f(2)=f(3)=f(4)$、$f(5)=f(6)=f(7)$ 以及 $f(8)=f(9)$,亦可通过如下算式计算 $sum(A)$;

\begin{equation*}
  sum(A)=f(0)\times 2+f(2)\times 3+f(5)\times 3 + f(8)\times 2
\end{equation*}

\examplebox{\lstinputlisting[frame=none]{data/24/1-2.in}}{\lstinputlisting[frame=none]{data/24/1-2.out}}

\subsection*{子任务}

$50$ \% 的测试数据满足 $1\le n\le 200$ 且 $n\le N\le 1000$;

全部的测试数据满足 $1\le n\le 200$ 且 $n\le N\le 10^7$。

\subsection*{提示}

若存在区间 $[i,j)$ 满足 $f(i)=f(i+1)=\cdots=f(j-1)$,使用乘法运算 $f(i)\times (j-i)$ 代替将 $f(i)$ 到 $f(j-1)$ 逐个相加,或可大幅提高算法效率。
