\subsection*{题目背景}

上一题“序列查询”中说道:
$A=[A_0,A_1,A_2,\cdots,A_n]$ 是一个由 $n+1$ 个 $[0,N)$ 范围内整数组成的序列,满足 $0 = A_0 < A_1 < A_2 < \cdots < A_n < N$
。基于序列 $A$,对于 $[0,N)$ 范围内任意的整数 $x$,查询 $f(x)$ 定义为:序列 $A$ 中{\heiti{小于等于}} $x$ 的整数里最大的数的下标。

对于给定的序列 $A$ 和整数 $x$,查询 $f(x)$ 是一个很经典的问题,可以使用二分搜索在 $\mathbf{O}(\log n)$ 的时间复杂度内轻松解决。但在 IT 部门讨论如何实现这一功能时,小 P 同学提出了些新的想法。

\subsection*{题目描述}

小 P 同学认为,如果事先知道了序列 $A$ 中整数的分布情况,就能直接估计出其中小于等于 $x$ 的最大整数的大致位置。接着从这一估计位置开始线性查找,锁定 $f(x)$。如果估计得足够准确,线性查找的时间开销可能比二分查找算法更小。

比如说,如果 $A_1,A_2,\cdots,A_n$
 均匀分布在 $(0,N)$ 的区间,那么就可以估算出:
 
\begin{equation*}
    f(x)\approx \frac{(n+1)\cdot x}{N}
\end{equation*}

为了方便计算,小 P 首先定义了比例系数 $r=\lfloor \frac{N}{n+1} \rfloor$
 
,其中 $\lfloor \rfloor$ 表示下取整,即 $r$ 等于 $N$ 除以 $n+1$ 的商。进一步地,小 P 用 $g(x)=\lfloor \frac{x}{r}\rfloor$
 
 表示自己估算出的 $f(x)$ 的大小,这里同样使用了下取整来保证 $g(x)$ 是一个整数。

显然,对于任意的询问 $x\in [0,N)$,$g(x)$ 和 $f(x)$ 越接近则说明小 P 的估计越准确,后续进行线性查找的时间开销也越小。因此,小 P 用两者差的绝对值 $|g(x)-f(x)|$ 来表示处理询问 $x$ 时的误差。

为了整体评估小 P 同学提出的方法在序列 $A$ 上的表现,试计算:
 
\begin{equation*}
    error(A)=\sum\limits_{i=0}^{N-1}{|g(i)-f(i)|}=|g(0)-f(0)| + \cdots + |g(N-1)-f(N-1)|
\end{equation*}

\subsection*{输入格式}

从标准输入读入数据。

输入的第一行包含空格分隔的两个正整数 $n$ 和 $N$。

输入的第二行包含 $n$ 个用空格分隔的整数 $A_1,A_2,\cdots,A_n$
。

注意 $A_0$
 固定为 $0$,因此输入数据中不包括 $A_0$
。

\subsection*{输出格式}

输出到标准输出。

仅输出一个整数,表示 $error(A)$ 的值。

\examplebox*{\lstinputlisting[frame=none]{data/24/2-1.in}}{\lstinputlisting[frame=none]{data/24/2-1.out}}

$A=[0, 2, 5, 8]$

$r = \lfloor \frac{N}{n+1}\rfloor=\lfloor \frac{10}{3+1}\rfloor=2$

\begin{table}[H]
  \centering
  \begin{tabular}{ccccccccccc}
    \toprule
    % \midrule
    $i$ & $0$ & $1$ & $2$ & $3$ & $4$ & $5$ & $6$ & $7$ & $8$ & $9$ \\
    $f(i)$ & $0$ & $0$ & $1$ & $1$ & $1$ & $2$ & $2$ & $2$ & $3$ & $3$ \\
    $g(i)$ & $0$ & $0$ & $1$ & $1$ & $2$ & $2$ & $3$ & $3$ & $4$ & $4$ \\
    $|g(i)-f(i)|$ & $0$ & $0$ & $0$ & $0$ & $1$ & $0$ & $1$ & $1$ & $1$ & $1$ \\
    \bottomrule
  \end{tabular}
\end{table}

\examplebox{\lstinputlisting[frame=none]{data/24/2-2.in}}{\lstinputlisting[frame=none]{data/24/2-2.out}}

\examplebox*{\lstinputlisting[frame=none]{data/24/2-3.in}}{\lstinputlisting[frame=none]{data/24/2-3.out}}

$A=[0, 1, 3]$

$r = \lfloor \frac{N}{n+1}\rfloor=\lfloor \frac{10}{2+1}\rfloor=3$

\begin{table}[H]
  \centering
  \begin{tabular}{ccccccccccc}
    \toprule
    % \midrule
    $i$ & $0$ & $1$ & $2$ & $3$ & $4$ & $5$ & $6$ & $7$ & $8$ & $9$ \\
    $f(i)$ & $0$ & $1$ & $1$ & $2$ & $2$ & $2$ & $2$ & $2$ & $2$ & $2$ \\
    $g(i)$ & $0$ & $0$ & $0$ & $1$ & $1$ & $1$ & $2$ & $2$ & $2$ & $3$ \\
    $|g(i)-f(i)|$ & $0$ & $1$ & $1$ & $1$ & $1$ & $1$ & $0$ & $0$ & $0$ & $1$ \\
    \bottomrule
  \end{tabular}
\end{table}

\subsection*{子任务}

$70$ \% 的测试数据满足 $1\le n\le 200$ 且 $n\le N\le 1000$;

全部的测试数据满足 $1\le n\le 10^5$ 且 $n\le N\le 10^9$。

\subsection*{提示}

需要注意,输入数据 $[A_1\cdots A_n]$
 并不一定均匀分布在 $(0,N)$ 区间,因此总误差 $error(A)$ 可能很大。