\section{202112-2 序列查询新解}

\subsection*{题目背景}

上一题“序列查询”中说道:
$A=[A_0,A_1,A_2,\cdots,A_n]$ 是一个由 $n+1$ 个 $[0,N)$ 范围内整数组成的序列,满足 $0 = A_0 < A_1 < A_2 < \cdots < A_n < N$
。基于序列 $A$,对于 $[0,N)$ 范围内任意的整数 $x$,查询 $f(x)$ 定义为:序列 $A$ 中{\heiti{小于等于}} $x$ 的整数里最大的数的下标。

对于给定的序列 $A$ 和整数 $x$,查询 $f(x)$ 是一个很经典的问题,可以使用二分搜索在 $\mathbf{O}(\log n)$ 的时间复杂度内轻松解决。但在 IT 部门讨论如何实现这一功能时,小 P 同学提出了些新的想法。

\subsection*{题目描述}

小 P 同学认为,如果事先知道了序列 $A$ 中整数的分布情况,就能直接估计出其中小于等于 $x$ 的最大整数的大致位置。接着从这一估计位置开始线性查找,锁定 $f(x)$。如果估计得足够准确,线性查找的时间开销可能比二分查找算法更小。

比如说,如果 $A_1,A_2,\cdots,A_n$
 均匀分布在 $(0,N)$ 的区间,那么就可以估算出:
 
\begin{equation*}
    f(x)\approx \frac{(n+1)\cdot x}{N}
\end{equation*}

为了方便计算,小 P 首先定义了比例系数 $r=\lfloor \frac{N}{n+1} \rfloor$
 
,其中 $\lfloor \rfloor$ 表示下取整,即 $r$ 等于 $N$ 除以 $n+1$ 的商。进一步地,小 P 用 $g(x)=\lfloor \frac{x}{r}\rfloor$
 
 表示自己估算出的 $f(x)$ 的大小,这里同样使用了下取整来保证 $g(x)$ 是一个整数。

显然,对于任意的询问 $x\in [0,N)$,$g(x)$ 和 $f(x)$ 越接近则说明小 P 的估计越准确,后续进行线性查找的时间开销也越小。因此,小 P 用两者差的绝对值 $|g(x)-f(x)|$ 来表示处理询问 $x$ 时的误差。

为了整体评估小 P 同学提出的方法在序列 $A$ 上的表现,试计算:
 
\begin{equation*}
    error(A)=\sum\limits_{i=0}^{N-1}{|g(i)-f(i)|}=|g(0)-f(0)| + \cdots + |g(N-1)-f(N-1)|
\end{equation*}

\subsection*{输入格式}

从标准输入读入数据。

输入的第一行包含空格分隔的两个正整数 $n$ 和 $N$。

输入的第二行包含 $n$ 个用空格分隔的整数 $A_1,A_2,\cdots,A_n$
。

注意 $A_0$
 固定为 $0$,因此输入数据中不包括 $A_0$
。

\subsection*{输出格式}

输出到标准输出。

仅输出一个整数,表示 $error(A)$ 的值。

\examplebox*{\lstinputlisting[frame=none]{data/24/2-1.in}}{\lstinputlisting[frame=none]{data/24/2-1.out}}

$A=[0, 2, 5, 8]$

$r = \lfloor \frac{N}{n+1}\rfloor=\lfloor \frac{10}{3+1}\rfloor=2$

\begin{table}[H]
  \centering
  \begin{tabular}{ccccccccccc}
    \toprule
    % \midrule
    $i$ & $0$ & $1$ & $2$ & $3$ & $4$ & $5$ & $6$ & $7$ & $8$ & $9$ \\
    $f(i)$ & $0$ & $0$ & $1$ & $1$ & $1$ & $2$ & $2$ & $2$ & $3$ & $3$ \\
    $g(i)$ & $0$ & $0$ & $1$ & $1$ & $2$ & $2$ & $3$ & $3$ & $4$ & $4$ \\
    $|g(i)-f(i)|$ & $0$ & $0$ & $0$ & $0$ & $1$ & $0$ & $1$ & $1$ & $1$ & $1$ \\
    \bottomrule
  \end{tabular}
\end{table}

\examplebox{\lstinputlisting[frame=none]{data/24/2-2.in}}{\lstinputlisting[frame=none]{data/24/2-2.out}}

\examplebox*{\lstinputlisting[frame=none]{data/24/2-3.in}}{\lstinputlisting[frame=none]{data/24/2-3.out}}

$A=[0, 1, 3]$

$r = \lfloor \frac{N}{n+1}\rfloor=\lfloor \frac{10}{2+1}\rfloor=3$

\begin{table}[H]
  \centering
  \begin{tabular}{ccccccccccc}
    \toprule
    % \midrule
    $i$ & $0$ & $1$ & $2$ & $3$ & $4$ & $5$ & $6$ & $7$ & $8$ & $9$ \\
    $f(i)$ & $0$ & $1$ & $1$ & $2$ & $2$ & $2$ & $2$ & $2$ & $2$ & $2$ \\
    $g(i)$ & $0$ & $0$ & $0$ & $1$ & $1$ & $1$ & $2$ & $2$ & $2$ & $3$ \\
    $|g(i)-f(i)|$ & $0$ & $1$ & $1$ & $1$ & $1$ & $1$ & $0$ & $0$ & $0$ & $1$ \\
    \bottomrule
  \end{tabular}
\end{table}

\subsection*{子任务}

$70$ \% 的测试数据满足 $1\le n\le 200$ 且 $n\le N\le 1000$;

全部的测试数据满足 $1\le n\le 10^5$ 且 $n\le N\le 10^9$。

\subsection*{提示}

需要注意,输入数据 $[A_1\cdots A_n]$
 并不一定均匀分布在 $(0,N)$ 区间,因此总误差 $error(A)$ 可能很大。

\subsection{与上一题的比较}

\begin{enumerate}
    \item 上一题是求和,而本题要求求绝对值的和,无法转化为两者求差的形式。
    \item $f(x),g(x)$ 的变化是各自独立的,当 $f(x)$ 改变时,$g(x)$ 可能不变,也可能改变;$g(x)$ 对 $f(x)$ 也是如此。
    \item 对于所有数据点,$n$ 和 $N$ 都增大了许多。如果复杂度涉及到 $n$,则最多预计为 $\mathbf{O}(n\log n)$ 级别;如果涉及到 $N$,则必须是亚线性级别。
\end{enumerate}

\subsection{$70$\% 数据——计算出每个 $f(x),g(x)$ 的值}

\subsubsection{思路}

由于1,2条限制,我们无法直接对 $f(x),g(x)$ 分别进行处理。但我们可以求出每个 $f(x),g(x)$ 的值,再计算求和即可。

$f(x)$ 的计算同第一问,任意方法皆可。单个 $g(x)$ 的值可以直接 $\mathbf{O}(1)$ 求得。

\subsubsection{C++实现}

待补充

\subsection{$100$\% 数据——对 $f(x),g(x)$ 都相同的区间进行求和处理}

\subsubsection{思路}

注:为了防止混淆,将题目中的 $r$ 改为 $ratio$。

假设 $f(x)$ 一共有 $x$ 种取值,$g(x)$ 一共有 $y$ 种取值。
直接来看 $f(x),g(x)$ 的组合一共有 $xy$ 种,
但注意到 $f(x),g(x)$ 都是单调不递减函数,所以真正的组合只有 $x+y$ 种。

在第一题中已经说明 $f(x)$ 的取值范围为 $[0,n]$,在 $\mathbf{O}(n)$ 级别。
考虑 $g(x)$ 的取值情况,将 $ratio$ 的公式带入可以得到 $g(x)=\lfloor \frac{x}{ratio}\rfloor=\lfloor\frac{x}{\lfloor \frac{N}{n+1}\rfloor}\rfloor$。
由于 $x$ 取值有 $N$ 种,所以 $g(x)$ 的取值是 $\mathbf{O}(\frac{N}{\frac{N}{n+1}})=\mathbf{O}(n)$ 级别的。
所以,整体复杂度为 $\mathbf{O}(n+n)=\mathbf{O}(n)$。

\begin{note}
有些时候,题目给出的某些量的值会比较特殊(如本题 $ratio=\lfloor\frac{N}{n+1}\rfloor$),
代表着出题人可能想要隐藏某些做法,但不得不为了让时间复杂度正确而妥协。
在没有思路的时候,可以作为突破口。
\end{note}

考虑范围问题:假设当前左端点为 $l$,如何找到右端点 $r$,满足 $f(l)=f(l+1)=\cdots=f(r),g(l)=g(l+1)\cdots=g(r)$ 且 $f(l)\not=f(r+1)\ or\ g(l)\not=g(r+1)$。
我们可以对 $f(x),g(x)$ 分别考虑:

\begin{enumerate}
    \item 对于 $f(x)$ 而言,第一个满足 $f(x)=f(l)+1$ 的 $x$ 值为 $A_{f_l + 1}$。
    \item 对于 $g(x)$ 而言,因为分母 $ratio$ 是固定的,所以值相同的区间长度也是固定为 $ratio$。
    我们不妨将 $g(x)$ 值相同的数字为一组,则可以得到 $[0,ratio-1],[ratio,2\cdot ratio-1],\cdots,[n\cdot ratio,(n+1)\cdot ratio-1],\cdots$ 
    这样的分组序列,每组的 $g(x)$ 取值为 $0,1,\cdots,n,\cdots$。
    可以发现,对于一个数 $l$,其所属的分组是 $\lfloor \frac{l}{ratio}\rfloor$,也即 $g(l)$;
    而下一组开始的第一个数为 $ratio\cdot (g(l)+1)$,从而可以得到右端点 $r = ratio\cdot (g(l)+1) - 1$。
    \item 在 $f(x),g(x)$ 计算得到的右端点中,选择较小的一个作为计算的右端点。
\end{enumerate}

计算完一段后,设 $l=r+1$ 继续计算下一段,直到结束。时间复杂度 $\mathbf{O}(n)$。

\subsubsection{C++实现}

\lstinputlisting[language=c++]{code/24/202112-2-100.cpp}

\subsection{$100$\% 思路——以 $f(x)$ 为单位,讨论内部 $g(x)$ 求和}

感谢 \href{https://github.com/DoctorLazy}{DoctorLazy} 
提供的宝贵思路,原文可以查看 \href{https://github.com/DoctorLazy}{第二题100分题解 by DoctorLazy.md}
。

\subsubsection{思路}

和上文同样的思路,我们需要进行区间求和来降低复杂度。
一种思路是,整体上对 $f(x)$ 进行求和,而在内部对 $g(x)$ 的情况进行分类讨论。

我们单独考虑每一个 $f(x)$ 的区间,每个区间上 $f(x)$ 的值相同。
可以观察到,对于一个区间上的下标 $i$,可能存在 $g(i)\ge f(i)$,也可能存在 $g(i)<f(i)$。
求绝对值时,前者用 $g(x)-f(x)$,后者用 $f(x)-g(x)$。

观察到,由于 $g(x)$ {\heiti{单调不减}}的性质,我们可以得到:
对于该区间,一定存在一个下标 $p$,如同一个分界线,
当 $i\ge p$ 时,有 $g(i)\ge f(i)$,
当 $i<p$,有 $g(i)<f(i)$。
这样,就把该区间分成了两个“小区间”。
我们就可以用“乘法思想”来加速两个“小区间”的求解了。

更规范些,用 $contribution(i)$ 代表区间 $[A_i, A_{i+1})$ 对答案的贡献,
用 $len(l, r) = r - l + 1$ 代表区间长度, 
用公式可以表达为:

\begin{align*}
contribution(i) & =len(A_i,p-1)\times f(x)-\sum_{x=A[i]}^{p-1}g(x) \\
& +\sum_{x=p}^{A_{i+1}-1}g(x)- len(p,A_{i+1}-1)\times f(x) \\
\end{align*}

上式中,$f(x)$ 是一个常数,所以乘以“小区间”的长度即可;
$g(x)$ 的求和,大家可以发挥数学思维:因为 $g(x)$ 其实非常规律,它的每一块是定长的,我们可以通过除法和取余来确定相同值的数量,再利用乘法思想求和,灵活实现,在 $\mathbf{O}(n)$ 时间内求出即可。
$p$ 的具体值可以通过在 $g(x)$ 中二分查找,$\mathbf{O}(\log n)$ 时间内求出,$n$ 为区间的长度。

一个例子:

\begin{table}[H]
\centering
\begin{tabular}{ccccccc}
    \toprule
    % \midrule
    $x$ & $\cdots$ & 4 & 5 & 6 & 7 & $\cdots$ \\
    $f(x)$ & $\cdots$ & 2 & 2 & 2 & 2 & $\cdots$ \\
    $g(x)$ & $\cdots$ & 1 & 1 & 2 & 2 & $\cdots$ \\
    \bottomrule
\end{tabular}
\end{table}

上面的表格截取了一个小区间,
$f(x)$ 的值固定 $2$,$p=6$,那么 $p$ 的左边用 $f(x)-g(x)$,$p$ 的右边用 $g(x)-f(x)$。

当然,有一个特殊的边界情况,
那就是该区间上有可能所有的 $g(x)$ 都绝对大于或小于 $f(x)$,这时候 $p$ 可能会在区间外。
该情况大家可以对 $p$ 设置初值,然后在写完二分后加以判断即可。

\subsubsection{C++实现}

\lstinputlisting[language=c++]{code/24/202112-2-100-2.cpp}