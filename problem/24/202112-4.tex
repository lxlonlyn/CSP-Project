% \section{202112-4 磁盘文件操作}

% \subsection*{题目背景}

小C对计算机运行的原理很感兴趣,经常进行一些研究和实验。

有一天,他在尝试删除一个好几GB大小的文件时,惊奇地发现删除操作几乎在一瞬间就完成了!
这让他很是纳闷:如果计算机在每次删除文件时都直接在磁盘上把对应的数据抹掉,不是应该要花挺长时间吗?

于是他找来了小S和小P一起讨论。
小S说,或许计算机是一个很“懒”的体系,在删除时不会真的去抹除数据吧?
而小P则更见多识广一些,他当即找来了一个号称能“恢复磁盘数据”的软件,当场把小C刚刚删除的文件恢复了!

这让小C有了更强的好奇心,于是他们决定设计一个模型来模拟一个磁盘文件的写入、删除及恢复过程。
但是在他们生活的西西艾弗岛上没有合适的条件来运行他们的模型,于是他们联系了带着一台算力超强的电脑来西西艾弗岛旅游的你来帮助他们。

\subsection*{题目描述}

在小C、小S和小P设计的模型中,计算机中有 $n$ 段程序(编号为 $1\sim n$),它们共享一块大小为 $m$ 的磁盘空间(编号为 $1\sim m$ ),磁盘上的每个位置可以写入一个整数。

最初,磁盘上每个位置上的数都是 $0$,并不被任何程序占用。

现在,这 $n$ 段程序同时执行,在某一时刻,某段程序可能对磁盘数据进行读写等操作。

操作共 $k$ 个,按时间先后顺序给出,具体操作如下:

\begin{itemize}
    \item $0\ id\ l\ r\ x$:编号为 $id$ 的程序尝试向磁盘空间中 $[l,r]$ 位置上每个位置都写入一个整数 $x$。
    \begin{itemize}
        \item 操作执行过程中,程序 $id$ 会尝试从最左端 $l$ 开始向右顺次写入数据。
        \item 对于每个位置,若目前不被任何程序占用,则成功写入整数 $x$,并将其视为被程序 $id$ 占用;
        \item 若该位置目前正被程序 $id$ 自己占用,则这次写入的 $x$ 可以覆盖之前写入的结果,此后该位置仍被程序 $id$ 占用;
        \item 直到成功向 $r$ 位置写入数据,或遇到第一个正在被其他程序占用的位置为止,此时该操作立刻中断。
    \end{itemize}
    \item $1\ id\ l\ r$:程序 $id$ 尝试删除磁盘中 $[l,r]$ 位置上的所有数据。
    \begin{itemize}
        \item 这一操作当且仅当 $[l,r]$ 区间内所有位置都正在被程序 $id$ 占用时才能成功执行。
        \item 执行效果为将其中所有位置都解除占用,即恢复到可以被任意程序写入的状态。{\heiti{但为了便于恢复数据,不会立即将全部位置重新覆盖成 $0$}}。
        \item 否则,认为此操作执行失败,不进行任何修改。
    \end{itemize}
    \item $2\ id\ l\ r$:程序 $id$ 尝试恢复磁盘中 $[l,r]$ 位置上的所有数据。
    \begin{itemize}
        \item 这一操作当且仅当 $[l,r]$ 区间内所有位置都未被占用,且{\heiti{上一次被占用是被程序 $id$ 占用}}时才能成功执行。
        \item 执行效果为将其中所有位置恢复为{\heiti{被程序 $id$ 占用}}的状态,同时由于之前删除操作并未改变其存储的值,因此本次操作也不需要改变每个位置上的值。
        \item 否则,认为此操作执行失败,不进行任何修改。
    \end{itemize}
    \item $3\ p$:尝试读取磁盘中 $p$ 位置的数据,返回结果为两个整数。
    \begin{itemize}
        \item 如果该位置当前正被程序 $id$ 占用且存储的值为 $p$,返回结果为 $id\ p$。
        \item 如果该位置当前没有被任何程序占用,返回 $0\ 0$。
    \end{itemize}
\end{itemize}

你需要实现一个程序,帮助小C、小S和小P来模拟实现上述过程,并对于每个操作输出操作结果。

\subsection*{输入格式}

从标准输入读入数据。

第一行:$3$ 个正整数 $n,m,k$。

接下来 $k$ 行,每行若干个整数描述一个操作,格式如上所述。

\subsection*{输出格式}

输出到标准输出。

输出共 $k$ 行,对于每个操作输出一行。

对于每个写入操作,输出一个整数表示此次操作写入成功的最右位置;特别地如果该操作一个位置也没有写入成功,输出 $-1$。

对于每个删除、恢复操作,若该操作成功,输出一个字符串 OK ,否则输出一个字符串 FAIL 。

对于每个读取操作,输出两个整数表示此次查询的结果。

\examplebox{\lstinputlisting[frame=none]{data/24/4-1.in}}{\lstinputlisting[frame=none]{data/24/4-1.out}}

\subsection*{子任务}

对于 $25$\% 的数据,$n,k\le 2000,m\le 10000$;

对于另外 $15$\% 的数据,没有删除、恢复操作;

对于另外 $20$\% 的数据,没有恢复操作;

对于另外 $15$\% 的数据,$n=1$。

对于 $100$\% 的数据,$1\le n,k\le 2\times 10^5,1\le m\le 10^9,1\le id\le n,1\le l\le r\le m,1\le p\le m,|x|\le 10^9$。

\subsection{$25\%$ 数据——直接模拟}

\subsubsection{思路}

我们按照题目要求进行对应操作即可,注意每一个要求执行的条件:

\begin{itemize}
    \item 写入操作:从左往右依次执行,直到第一个不被自己占用的位置。{\heiti{除了第一个点就被其他程序占用以外,必然会写入。}}遇到自己占用,则{\heiti{覆盖}}。
    \item 删除操作:同时整体进行,要求所有位置都被目前程序占用。{\heiti{要么全删,要么不做任何更改。}}
    \item 恢复操作:同时整体进行,要求所有位置都不被占用,且上次占用程序为目前程序。{\heiti{要么全恢复,要么不做任何更改。}}遇到自己占用,则{\heiti{不做任何更改}}。
    \item 读入操作:读取占用程序和数值,若未被占用,则输出 0 0。
\end{itemize}

\subsection{$100\%$ 数据——离散化+线段树}

\subsubsection{思路}

通过这道题的操作要求等,我们可以大致推测出这道题可能需要使用线段树。

\begin{note}
    如果没什么思路,可以拿各种数据结构往上套。
    例如本题,因为涉及区间修改、单点查询,对于树状数组来说负担太重,我们可以尝试其他数据结构。
    如果使用平衡树,则一般是要求出第 k 大数,或者是序列翻转类问题,对于本题来说不太契合。
    其他数据结构不再一一列举。综合考虑下,线段树是比较符合要求的。
\end{note}

\paragraph{考虑线段树的做法}

先不考虑线段树的内存空间问题,我们分析一下如何用线段树解决这道题目。

考虑我们需要维护的量,目前已知的有磁盘位置的值、目前占用程序 id、上次占用程序 id。

在这里,我们假设一个位置未被占用和被 id 为 0 的程序占用是等价的。

\begin{itemize}
    \item 写入操作:可以划分为{\heiti{找到写入右边界}}和{\heiti{直接写入}}两个操作。

          直接写入操作就是直接的线段树区间修改,
          而划分操作需要知道该区间{\heiti{被占用}}的位置是否属于将要写入的 id。
          我们不妨将这个量设为 id1。

    \item 删除操作:可以划分为{\heiti{判断是否可删}}和{\heiti{直接删除}}两个操作。

          直接删除操作就是直接的线段树区间修改,
          而判断是否可删需要知道该区间{\heiti{所有}}的位置是否属于将要写入的 id。
          我们不妨将这个量设为 id2,注意 id1 与 id2 的区别——是否允许包含未被占用的程序。

    \item 恢复操作:可以划分为{\heiti{判断是否可恢复}}和{\heiti{直接恢复}}两个操作。

          该操作与删除操作类似,不过需要注意的是判断时需要判断目前占用的 id 和上次被占用的 id。

    \item 读取操作:可以划分为{\heiti{查询占用程序 id}}和{\heiti{查询值}}两个操作。

          该操作是相对比较质朴的单点查询,当然也可以处理为区间。

\end{itemize}

通过以上分析,我们得到了需要维护的量:值、有关目前占用程序 id 的两个量、上次被占用的程序 id。我们考虑每个量针对父子之间的维护。

\begin{itemize}
    \item 值 val:每个节点代表取值的多少,若左右子节点不同则设为一个不存在的值。因为我们是单点查询,所以不用担心查询到不存在的值的问题。
    \item 被占用位置程序 id1:
          \begin{itemize}
              \item 若左右子节点都未被占用,则该节点标记为未占用;
              \item 若左右子节点中存在不唯一节点,则该节点标记为不唯一。
              \item 若左右子节点中一个节点未占用,则该节点标记为另一个非空节点的标记;
              \item 若左右子节点都非空且相等,则该节点标记为任意一个节点;
              \item 若左右子节点都非空且不等,则该节点标记为不唯一;
          \end{itemize}
    \item 被占用位置程序 id2:为了方便进行讨论,将未被程序占用节点视为被 id 为 0 的程序占用。
          \begin{itemize}
              \item 若左右子节点中存在不唯一节点,则该节点标记为不唯一。
              \item 若左右子节点相等,则该节点标记为任意一个节点;
              \item 若左右子节点不等,则该节点标记为不唯一;
          \end{itemize}
    \item 上一次被占用程序 lid:与 id2 相同。
          \begin{itemize}
              \item 若左右子节点中存在不唯一节点,则该节点标记为不唯一。
              \item 若左右子节点相等,则该节点标记为任意一个节点;
              \item 若左右子节点不等,则该节点标记为不唯一;
          \end{itemize}
\end{itemize}

\paragraph{解决空间问题}

理解线段树的解法之后,就会出现另一个问题:空间达到了 1e9 级别,肯定会 MLE。

我们可以从另一个角度考虑:一共有 $2\times 10^5$ 次询问,每次最多操作涉及一个区间,可以用两个端点表示。
考虑到临界处的影响,一次操作最多会涉及 4 个点
(比如原来的区间是 $[1, 10]$,我们更改了区间 $[3,5]$,那么得到的区间为 $[1,2],[3,5],[6,10]$,多出了 $2,3,5,6$ 四个点)。
那么总体来看,涉及到的点最多有 $2\times 10^5\times 4 = 8\times 10^5$ 个。

我们可以维持这些点的相对大小关系,而将其投影到一个值域较小的区域,就可以减少空间占用了。这种方法称为离散化。

\begin{definition}[离散化] \label{def:discretization}
    \hspace{2em}把无限空间中有限的个体映射到有限的空间中去,以此提高算法的时空效率。通俗的说,离散化是在不改变数据相对大小的条件下,对数据进行相应的缩小。
    离散化本质上可以看成是一种哈希,其保证数据在哈希以后仍然保持原来的全/偏序关系。

    \hspace{2em}离散化的步骤:
    \begin{enumerate}
        \item 统计所有出现过的数字,在知道确切上界的时候可以用数组,不清楚情况下可以用 vector;
        \item 对所有的数据排序 (sort)、去重 (unique);
        \item 对于每个数,其离散化后的对应值即为其在排序去重后数组中的位置,可以通过二分 (lower\_bound) 确定。
    \end{enumerate}

\end{definition}

这道题允许我们提前将所有可能出现的数记录下来(当然不是所有的题目都允许这样),所以这道题就解决了。
线段树节点的个数与询问个数成比例,时间复杂度 $\mathrm{O}(k\log k)$。

\subsubsection{C++实现}

\lstinputlisting[language=c++]{code/24/202112-4-100.cpp}

\subsection{$100\%$ 数据——动态开点线段树}

\subsubsection{思路}

在上一题的做法中,我们需要先读入所有的数据并进行离散化处理,之后再执行主要的算法过程。
但不是所有的题目都可以在执行主要的算法过程前得到所有的输入数据。

\begin{definition}[离线算法] \label{def:offline algorithm}
    \hspace{2em}要求在执行算法前输入数据已知的算法称为离线算法。
    一般而言,如果没有对输入输出做特殊处理,则可以用离线算法解决该问题。
\end{definition}

\begin{definition}[在线算法] \label{def:online algorithm}
    \hspace{2em}不需要输入数据已知就可以执行的算法称为在线算法。
    一般而言,如果对输入输出做特殊处理(如本次的询问需要与上次执行的答案进行异或才能得到真正的询问),则只能用离线算法解决该问题。

    \hspace{2em}对于一道能用离线和在线算法解决的题目,如果出题人对数据进行了加密处理,导致只能使用在线算法,则我们称这道题是{\heiti{强制在线}}的。
\end{definition}

离散化需要事先知道所有可能出现的数,所以是{\heiti{离线算法}}。如果要强制在线,就需要另一种思路。

同样,从询问涉及的点有限出发,我们考虑最多能涉及线段树上点的个数。
线段树的高度为 $\mathrm{O}(\log m)$,假设每个涉及查询的点都到达了线段树的叶子结点,
且不考虑根到任意两个结点之间重复的节点,则总共涉及的线段树节点数的个数为 $\mathrm{O}(k\log m)$。
所以我们只需要为用到的节点开辟空间即可。

针对一般的线段树,我们是预先建好了整棵线段树(build 函数),
每个线段树节点的左右子节点编号与其本身编号都是对应的(通常一个子节点是父结点的二倍,而另一个子节点则相差 1)。
而对于这种只为需要用到节点开辟空间的线段树,其左右子树只有在需要的时候才会被创建,
所以编号间没有特定关系,需要记录。

考虑什么时候需要开辟新结点:在初始化的时候需要开创一个根节点;
在进行修改及查询的时候,如果区间不是所要的区间,则需要开创新的节点。
有一个技巧是,在修改和查询的时候往往要下传标记(pushdown),可以在此之前检查是否需要开创节点。

\subsubsection{C++实现}

\lstinputlisting[language=c++]{code/24/202112-4-100-2.cpp}

