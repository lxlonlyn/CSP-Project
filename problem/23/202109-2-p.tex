\subsection*{题目描述}

$A_1, A_2, \cdots, A_n$ 是一个由 $n$ 个自然数(非负整数)组成的数组。
我们称其中 $A_i, \cdots, A_j$ 是一个非零段,当且仅当以下条件同时满足:

\begin{itemize}
    \item $1 \le i \le j \le n$;
    \item 对于任意的整数 $k$,若 $i \le k \le j$,则 $A_k > 0$;
    \item $i = 1$ 或 $A_{i-1} = 0$;
    \item $j = n$ 或 $A_{j+1} = 0$。
\end{itemize}

下面展示了几个简单的例子:

\begin{itemize}
    \item $A = [3, 1, 2, 0, 0, 2, 0, 4, 5, 0, 2]$ 中的 $4$ 个非零段依次为 $[3, 1, 2]$、$[2]$、$[4, 5]$ 和 $[2]$;
    \item $A = [2, 3, 1, 4, 5]$ 仅有 $1$ 个非零段;
    \item $A = [0, 0, 0]$ 则不含非零段(即非零段个数为 $0$)。
\end{itemize}

现在我们可以对数组 $A$ 进行如下操作:
任选一个正整数 $p$,然后将 $A$ 中所有小于 $p$ 的数都变为 $0$。
试选取一个合适的 $p$,使得数组 $A$ 中的非零段个数达到最大。
若输入的 $A$ 所含非零段数已达最大值,可取 $p=1$,即不对 $A$ 做任何修改。

\subsection*{输入格式}

从标准输入读入数据。

输入的第一行包含一个正整数 $n$。

输入的第二行包含 $n$ 个用空格分隔的自然数 $A_1, A_2, \cdots, A_n$。

\subsection*{输出格式}

输出到标准输出。

仅输出一个整数,表示对数组 $A$ 进行操作后,其非零段个数能达到的最大值。

\subsection*{样例}

输入\#1:

\begin{lstlisting}
11
3 1 2 0 0 2 0 4 5 0 2
\end{lstlisting}

输出\#1:

\begin{lstlisting}
5
\end{lstlisting}

解释\#1:

$p = 2$ 时,$A = [3, 0, 2, 0, 0, 2, 0, 4, 5, 0, 2]$,$5$ 个非零段依次为 $[3]$、$[2]$、$[2]$、$[4, 5]$ 和 $[2]$;此时非零段个数达到最大。

输入\#2:

\begin{lstlisting}
14
5 1 20 10 10 10 10 15 10 20 1 5 10 15
\end{lstlisting}

输出\#2:

\begin{lstlisting}
4
\end{lstlisting}

解释\#2:

$p = 12$ 时,$A = [0, 0, 20, 0, 0, 0, 0, 15, 0, 20, 0, 0, 0, 15]$,$4$ 个非零段依次为 $[20]$、$[15]$、$[20]$ 和 $[15]$;此时非零段个数达到最大。

输入\#3:

\begin{lstlisting}
3
1 0 0
\end{lstlisting}

输出\#3:

\begin{lstlisting}
1
\end{lstlisting}

解释\#3:

$p = 1$ 时,$A = [1, 0, 0]$,此时仅有 $1$ 个非零段 $[1]$,非零段个数达到最大。

输入\#4:

\begin{lstlisting}
3
0 0 0
\end{lstlisting}

输出\#4:

\begin{lstlisting}
0
\end{lstlisting}

解释\#4:

无论 $p$ 取何值,$A$ 都不含有非零段,故非零段个数至多为 $0$。

\subsection*{子任务}

$70\%$ 的测试数据满足 $n \le 1000$;

全部的测试数据满足 $n \le 5 \times 10^{5}$,且数组 $A$ 中的每一个数均不超过 $10^{4}$。


