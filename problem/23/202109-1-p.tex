\subsection*{题目描述}

$A_1, A_2, \cdots, A_n$ 是一个由 $n$ 个{\heiti{自然数}}(即非负整数)组成的数组。
在此基础上,我们用数组 $B_1 \cdots B_n$ 表示 $A$ 的前缀最大值。

\begin{equation*}
    B_i = \max \left\{ A_1, A_2,  \cdots, A_i \right\}
\end{equation*}

如上所示,$B_i$ 定义为数组 $A$ 中前 $i$ 个数的最大值。
根据该定义易知 $A_1 = B_1$,且随着 $i$ 的增大,$B_i$ 单调不降。
此外,我们用 $sum = A_1 + A_2 + \cdots + A_n$ 表示数组 $A$ 中 $n$ 个数的总和。

现已知数组 $B$,我们想要根据 $B$ 的值来反推数组 $A$。
显然,对于给定的 $B$,$A$ 的取值可能并不唯一。
试计算,在数组 $A$ 所有可能的取值情况中,$sum$ 的最大值和最小值分别是多少?

\subsection*{输入格式}

从标准输入读入数据。

输入的第一行包含一个正整数 $n$。

输入的第二行包含 $n$ 个用空格分隔的自然数 $B_1, B_2, \cdots, B_n$。

\subsection*{输出格式}

输出到标准输出。

输出共两行。

第一行输出一个整数,表示 $sum$ 的最大值。

第二行输出一个整数,表示 $sum$ 的最小值。

\subsection*{样例}

输入\#1:

\begin{lstlisting}
6
0 0 5 5 10 10
\end{lstlisting}

输出\#1:

\begin{lstlisting}
30
15
\end{lstlisting}

解释\#1:

数组 $A$ 的可能取值包括但不限于以下三种情况。

\begin{itemize}
    \item 情况一:$A = [0, 0, 5, 5, 10, 10]$
    \item 情况二:$A = [0, 0, 5, 3, 10, 4]$
    \item 情况三:$A = [0, 0, 5, 0, 10, 0]$
\end{itemize}

其中第一种情况 $sum = 30$ 为最大值,第三种情况 $sum = 15$ 为最小值。

输入\#2:

\begin{lstlisting}
7
10 20 30 40 50 60 75
\end{lstlisting}

输出\#2:

\begin{lstlisting}
285
285
\end{lstlisting}

解释\#2:

$A = [10, 20, 30, 40, 50, 60, 75]$ 是唯一可能的取值,所以 $sum$ 的最大、最小值均为 $285$。

\subsection*{子任务}

$50\%$ 的测试数据满足数组 $B$ 单调递增,即 $0 < B_1 < B_2 < \cdots < B_n < 10^{5}$;

全部的测试数据满足 $n \le 100$ 且数组 $B$ 单调不降,即 $0 \le B_1 \le B_2 \le \cdots \le B_n \le 10^{5}$。


