\subsection*{题目背景}

\begin{quotation}
“你知道对长跑选手来说,最棒的赞美是什么吗?”

“是‘快’吗?”

“不,是‘强’,”清濑说,“光跑得快,是没办法在长跑中脱颖而出的。天候、场地、比赛的发展、体能,还有自己的精神状态——长跑选手必须冷静分析这许多要素,即使面对再大的困难,也要坚忍不拔地突破难关。长跑选手需要的,是真正的‘强’。所以我们必须把‘强’当作最高的荣誉,每天不断跑下去。”

不论阿走或其他房客,全都全神贯注地聆听清濑的话。

“看了你这三个月来的表现,我越来越相信自己没看错人,”清濑接着说,“你很有天分,也很有潜力。所以呢,阿走,你一定要更相信自己,不要急着想一飞冲天。变强需要时间,也可以说它永远没有终点。长跑是值得一生投入的竞赛,有些人即使老了,仍然没有放弃慢跑或马拉松运动。”

——三浦紫苑《强风吹拂》

\end{quotation}

箱根驿传(正式名称为东京箱根间往复大学驿传竞走)是日本一项在每年 1 月 2-3 日举行的驿站接力赛,由关东学生田径联盟主办,关东的每所高校都有机会参加。
在日本,箱根驿传是新年假期必看的比赛,许多家庭会一边吃年糕汤一边欣赏激烈的比赛。

今年,京都大学也想派出长跑队参加箱根驿传,田径部的长跑教练组织起一批预备役运动员,并开展了严苛的训练。

\subsection*{题目描述}

京都大学的训练一共会持续 $m$ 天,在训练过程中正式队员的名单可能发生变化。
简单起见,我们约定在且仅在第 $t (1 \le t \le m)$ 天结束时,会有以下三种事件之一发生:

\begin{enumerate}
    \item 有一个学生跑 $10\mathrm{km}$ 的速度达到了正式队员要求,教练将其作为最后一名纳入正式队员的名单中,这个学生的强度为 $x$;或者速度排名在最后一位的正式队员,由于速度过慢,而被从正式队员的名单中淘汰。
    \begin{itemize}
        \item 在训练过程中,我们假定队员的速度的相对排名不会发生变化,与强度无关。
        \item 严苛的教练制订了残酷的规则:被淘汰的学生虽然依然会跟大家一起训练,但将不能再次加入本年度参加箱根驿传的正式队员的名单中。
    \end{itemize}
    \item 由于近日的训练,第 $s$ 天结束时速度排名为 $l$ 至 $r$ 的选手的强度有了变化,变为此前的 $y$ 倍。
    \item 教练在深夜想知道近日训练的效果,于是他统计了第 $s$ 天结束时速度排名为 $l$ 至 $r$ 的选手目前(即第 $t$ 天结束时)强度的和。由于这个结果可能很大,方便起见我们只考虑其模 $p$ 的值。
\end{enumerate}
出于学生们的隐私考虑,事件日志有可能会被加密。

\subsection*{输入格式}

从标准输入读入数据。

第一行为三个用空格隔开的整数 $m$,$p$ 和 $T$。

如果 $T = 0$,事件 1 中 $x = x'$,事件 2 中 $y = y'$;
如果 $T = 1$,表示事件日志被加密了,事件 1 中 $x = x' \oplus A$,事件 2 中 $y = y' \oplus A$,其中 $\oplus$ 为按位异或运算,$A$ 为此前最后一次事件 3 所统计出的结果。
如果此前没有事件 3 发生,则 $A = 0$。

接下来 $m$ 行,第 $t$ 行表示在第 $t$ 天结束时发生的事件:

\begin{itemize}
    \item $1 \ x'$:表示事件 1 发生。若 $x > 0$,表示有一个强度为 $x$ 的学生作为最后一名纳入正式队员的名单;若 $x = 0$,表示排名在最后的正式队员被从名单中淘汰。保证有 $0 \le x' < 2^{30}$。
    \item $2 \ s \ l \ r \ y'$:表示事件 2 发生。保证有 $1 \le s \le t$,$1 \le l \le r \le n$,$0 \le y' < 2^{30}$,其中 $n$ 为在第 $s$ 天结束时正式队员的人数。
    \item $3 \ s \ l \ r$:表示事件 3 发生。保证有 $1 \le s \le t$,$1 \le l \le r \le n$,其中 $n$ 为在第 $s$ 天结束时正式队员的人数。
\end{itemize}

\subsection*{输出格式}

输出到标准输出。

对于每一个事件 3,输出一行一个数字,为其所统计出的结果。

\examplebox*{\lstinputlisting[frame=none]{data/23/5-1.in}}{\lstinputlisting[frame=none]{data/23/5-1.out}}

第 1 天结束时,有一个强度为 $7$ 的学生被列为正式队员,我们不妨称他为小津。此时正式队员名单依次为:小津。

第 2 天结束时,有一个强度为 $3$ 的学生被列为正式队员,我们不妨称他为城崎。此时正式队员名单依次为:小津、城崎。

第 3 天结束时,城崎被淘汰了。此时正式队员名单为:小津。

第 4 天结束时,有一个强度为 $4$ 的学生被列为正式队员,我们不妨称他为樋口清太郎。此时正式队员名单依次为:小津、樋口清太郎。

第 5 天结束时,由于近日的训练,第 4 天正式队员名单中第 1 至 2 个人——即小津和樋口清太郎——的强度乘了 2,所以,小津的强度达到了 $14$,樋口清太郎的强度达到了 $8$。

第 6 天结束时,教练统计了第 2 天正式队员名单中第 1 至 2 个人——即小津和城崎——当前的强度,小津的强度为 $14$,城崎的强度为 $3$,故统计结果为 $17$,模 $p$ 的值为 $7$。

第 7 天结束时,由于近日的训练,第 1 天正式队员名单中的第 1 个人——即小津——的强度乘了 3,所以,小津的强度达到了 $42$。

第 8 天结束时,教练统计了第 6 天正式队员名单中第 1 至 2 个人——即小津和樋口清太郎——当前的强度,小津的强度为 $42$,樋口清太郎的强度为 $8$,故统计结果为 $50$,模 $p$ 的值为 $0$。

\examplebox{见 data/23/5-2.in}{见 data/23/5-2.out}

\subsection*{子任务}

$1 \le m \le 3 \times 10^5, 2 \le p < 2^{30}, mode \in { 0, 1 }$

\begin{table}[htbp]
\centering
\begin{tabular}{ccc}
    \toprule
    测试点 & 特殊性质 & $mode$ \\
    \midrule
    1 & $m \le 5000$ & 1 \\
    2 & 事件 1 中 $x > 0$ & 1 \\
    3 & 没有事件 2 & 1 \\
    4 & 事件 1 $x$ 在 $0,1$ 中随机选取 & 0 \\
    5 & $r-l\le 10$ & 0 \\
    6 & $r-l\le 10$ & 1 \\
    7,8 & 无 & 0 \\
    9,10 & 无 & 1 \\
    \bottomrule
\end{tabular}
\end{table}