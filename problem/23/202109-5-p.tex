\subsection*{题目背景}

\begin{quotation}
“你知道对长跑选手来说,最棒的赞美是什么吗?”

“是‘快’吗?”

“不,是‘强’,”清濑说,“光跑得快,是没办法在长跑中脱颖而出的。天候、场地、比赛的发展、体能,还有自己的精神状态——长跑选手必须冷静分析这许多要素,即使面对再大的困难,也要坚忍不拔地突破难关。长跑选手需要的,是真正的‘强’。所以我们必须把‘强’当作最高的荣誉,每天不断跑下去。”

不论阿走或其他房客,全都全神贯注地聆听清濑的话。

“看了你这三个月来的表现,我越来越相信自己没看错人,”清濑接着说,“你很有天分,也很有潜力。所以呢,阿走,你一定要更相信自己,不要急着想一飞冲天。变强需要时间,也可以说它永远没有终点。长跑是值得一生投入的竞赛,有些人即使老了,仍然没有放弃慢跑或马拉松运动。”

——三浦紫苑《强风吹拂》

\end{quotation}

箱根驿传(正式名称为东京箱根间往复大学驿传竞走)是日本一项在每年 1 月 2-3 日举行的驿站接力赛,由关东学生田径联盟主办,关东的每所高校都有机会参加。
在日本,箱根驿传是新年假期必看的比赛,许多家庭会一边吃年糕汤一边欣赏激烈的比赛。

今年,京都大学也想派出长跑队参加箱根驿传,田径部的长跑教练组织起一批预备役运动员,并开展了严苛的训练。

\subsection*{题目描述}

京都大学的训练一共会持续 $m$ 天,在训练过程中正式队员的名单可能发生变化。
简单起见,我们约定在且仅在第 $t (1 \le t \le m)$ 天结束时,会有以下三种事件之一发生:

\begin{enumerate}
    \item 有一个学生跑 $10\mathrm{km}$ 的速度达到了正式队员要求,教练将其作为最后一名纳入正式队员的名单中,这个学生的强度为 $x$;或者速度排名在最后一位的正式队员,由于速度过慢,而被从正式队员的名单中淘汰。
    \begin{itemize}
        \item 在训练过程中,我们假定队员的速度的相对排名不会发生变化,与强度无关。
        \item 严苛的教练制订了残酷的规则:被淘汰的学生虽然依然会跟大家一起训练,但将不能再次加入本年度参加箱根驿传的正式队员的名单中。
    \end{itemize}
    \item 由于近日的训练,第 $s$ 天结束时速度排名为 $l$ 至 $r$ 的选手的强度有了变化,变为此前的 $y$ 倍。
    \item 教练在深夜想知道近日训练的效果,于是他统计了第 $s$ 天结束时速度排名为 $l$ 至 $r$ 的选手目前(即第 $t$ 天结束时)强度的和。由于这个结果可能很大,方便起见我们只考虑其模 $p$ 的值。
\end{enumerate}
出于学生们的隐私考虑,事件日志有可能会被加密。

\subsection*{输入格式}

从标准输入读入数据。

第一行为三个用空格隔开的整数 $m$,$p$ 和 $T$。

如果 $T = 0$,事件 1 中 $x = x'$,事件 2 中 $y = y'$;
如果 $T = 1$,表示事件日志被加密了,事件 1 中 $x = x' \oplus A$,事件 2 中 $y = y' \oplus A$,其中 $\oplus$ 为按位异或运算,$A$ 为此前最后一次事件 3 所统计出的结果。
如果此前没有事件 3 发生,则 $A = 0$。

接下来 $m$ 行,第 $t$ 行表示在第 $t$ 天结束时发生的事件:

\begin{itemize}
    \item $1 \ x'$:表示事件 1 发生。若 $x > 0$,表示有一个强度为 $x$ 的学生作为最后一名纳入正式队员的名单;若 $x = 0$,表示排名在最后的正式队员被从名单中淘汰。保证有 $0 \le x' < 2^{30}$。
    \item $2 \ s \ l \ r \ y'$:表示事件 2 发生。保证有 $1 \le s \le t$,$1 \le l \le r \le n$,$0 \le y' < 2^{30}$,其中 $n$ 为在第 $s$ 天结束时正式队员的人数。
    \item $3 \ s \ l \ r$:表示事件 3 发生。保证有 $1 \le s \le t$,$1 \le l \le r \le n$,其中 $n$ 为在第 $s$ 天结束时正式队员的人数。
\end{itemize}

\subsection*{输出格式}

输出到标准输出。

对于每一个事件 3,输出一行一个数字,为其所统计出的结果。

\subsection*{样例}

输入\#1:

\begin{lstlisting}
8 10 0
1 7
1 3
1 0
1 4
2 4 1 2 2
3 2 1 2
2 1 1 1 3
3 6 1 2
\end{lstlisting}

输出\#1:

\begin{lstlisting}
7
0
\end{lstlisting}

解释\#1:

第 1 天结束时,有一个强度为 $7$ 的学生被列为正式队员,我们不妨称他为小津。此时正式队员名单依次为:小津。

第 2 天结束时,有一个强度为 $3$ 的学生被列为正式队员,我们不妨称他为城崎。此时正式队员名单依次为:小津、城崎。

第 3 天结束时,城崎被淘汰了。此时正式队员名单为:小津。

第 4 天结束时,有一个强度为 $4$ 的学生被列为正式队员,我们不妨称他为樋口清太郎。此时正式队员名单依次为:小津、樋口清太郎。

第 5 天结束时,由于近日的训练,第 4 天正式队员名单中第 1 至 2 个人——即小津和樋口清太郎——的强度乘了 2,所以,小津的强度达到了 $14$,樋口清太郎的强度达到了 $8$。

第 6 天结束时,教练统计了第 2 天正式队员名单中第 1 至 2 个人——即小津和城崎——当前的强度,小津的强度为 $14$,城崎的强度为 $3$,故统计结果为 $17$,模 $p$ 的值为 $7$。

第 7 天结束时,由于近日的训练,第 1 天正式队员名单中的第 1 个人——即小津——的强度乘了 3,所以,小津的强度达到了 $42$。

第 8 天结束时,教练统计了第 6 天正式队员名单中第 1 至 2 个人——即小津和樋口清太郎——当前的强度,小津的强度为 $42$,樋口清太郎的强度为 $8$,故统计结果为 $50$,模 $p$ 的值为 $0$。

输入\#2:

\begin{lstlisting}
200 307854322 1
1 304192542
1 261749745
1 227234660
1 258761107
1 71490397
1 72584186
1 172113773
1 170623186
1 109308637
1 108383253
1 221430535
1 184520171
1 12820964
1 64943840
1 271383631
1 103269159
1 12002213
1 141551258
1 200255671
1 303679342
1 177153246
1 242934504
1 192722694
1 81041418
1 129449540
1 208869479
1 193883084
1 47265951
1 14844237
1 204331401
1 120715260
1 183356222
1 151061115
1 97645108
1 95770509
1 10891614
1 136365751
1 277592250
1 244161106
1 74405936
1 140365146
1 22587603
1 172441554
1 300179553
1 235367849
1 75467014
1 291045594
1 220071302
1 26967280
1 279868778
1 109902396
1 286509675
1 275417760
1 74253569
1 57318310
1 147462465
1 89999340
1 17784677
1 245244350
1 138709004
1 214478013
1 134244031
1 298548097
1 17276277
1 183802269
1 22366514
1 275904549
1 142230969
1 116156399
1 63581175
1 136336228
1 214860504
1 72329372
1 204231581
1 78276583
1 277642488
1 81760292
1 7831561
1 134535873
1 42237141
1 165620849
1 286362129
1 87388726
1 288617590
1 97675237
1 113222505
1 292912
1 98092392
1 257549905
1 180583994
1 244157382
1 117371320
1 304810612
1 148813285
1 150599985
1 229632823
1 246806551
1 297736161
1 66536628
1 70165839
1 31086027
1 0
1 46984478
1 0
3 13 2 11
1 134407869
1 134407869
2 73 7 31 130418473
1 134407869
1 62804642
1 134407869
1 134407869
3 28 12 21
1 173819539
3 93 34 83
3 68 5 38
1 95422722
2 85 54 85 89788932
2 28 2 25 251954506
1 185543612
1 34466375
1 185543612
2 28 2 13 51844756
3 43 22 33
3 97 5 87
1 53569742
1 83590412
1 53569742
2 1 1 1 131620724
2 128 5 72 41971821
1 53569742
1 197197823
1 333156690
3 59 48 50
1 224641252
1 24037560
3 54 26 40
2 91 7 37 58105019
2 59 4 50 254285874
3 112 17 63
3 103 53 54
3 3 1 3
3 4 4 4
2 121 68 76 258392700
2 88 55 59 10180251
2 110 45 76 125533148
1 160394017
1 170735200
3 51 18 34
3 136 43 64
1 49346652
1 114223193
3 62 17 26
3 57 8 40
1 278848254
1 278848254
1 411603847
1 278848254
1 278848254
1 278848254
2 112 70 79 437553006
2 120 4 89 428224488
3 53 50 51
2 109 36 81 205513848
3 81 31 46
3 78 45 64
2 107 19 28 8661353
3 36 21 33
2 28 4 28 227981470
2 42 18 29 7067955
1 1536382
1 214941299
1 176373062
2 172 26 74 241772251
1 21368911
2 61 50 61 37266210
3 84 30 48
3 31 8 24
1 297156062
1 328205831
2 175 57 77 408989526
1 528105214
2 30 4 24 416690156
3 9 4 8
1 31664726
3 144 40 67
1 276474902
2 115 56 72 15951722
2 187 6 50 38653155
1 17582098
1 276474902
1 415944350
3 34 1 24
1 204461450
2 136 1 59 92444637
3 79 7 79
3 22 2 17
1 284506459
2 166 78 86 22058413
1 30931926
\end{lstlisting}

输出\#2:

\begin{lstlisting}
134407869
184056088
13293385
185543612
34566045
53569742
224641252
2971977
10689074
196073568
220349662
170735200
273278086
25664733
155812556
278848254
238098134
256392602
172981220
1536382
57033232
297156062
31664726
276474902
204461450
168966052
30931926
\end{lstlisting}

\subsection*{子任务}

$1 \le m \le 3 \times 10^5, 2 \le p < 2^{30}, mode \in { 0, 1 }$

\begin{table}[htbp]
\centering
\begin{tabular}{ccc}
    \toprule
    测试点 & 特殊性质 & $mode$ \\
    \midrule
    1 & $m \le 5000$ & 1 \\
    2 & 事件 1 中 $x > 0$ & 1 \\
    3 & 没有事件 2 & 1 \\
    4 & 事件 1 $x$ 在 $0,1$ 中随机选取 & 0 \\
    5 & $r-l\le 10$ & 0 \\
    6 & $r-l\le 10$ & 1 \\
    7,8 & 无 & 0 \\
    9,10 & 无 & 1 \\
    \bottomrule
\end{tabular}
\end{table}