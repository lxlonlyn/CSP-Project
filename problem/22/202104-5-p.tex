\subsection*{问题描述}

X 市最近生产了一批疫苗,需要运往各地使用。疫苗的运输是一个困难的问题:既要实现尽快时间送达,又要保证全程冷链,否则疫苗会损坏。

X 市的物流系统并不发达,只有 $n$ 个{\heiti{物流站点}}(以下简称“站点”)和 $m$ 条{\heiti{物流线路}}(以下简称“线路”),且该物流系统具有以下几个特点:

\begin{enumerate}

    \item 每条线路都是环线。即,从某个站点出发,经过一系列不重复的站点,最终回到出发站点。

    \item 每条线路上有且仅有一辆{\heiti{运输车}},以固定的时刻表(相邻站间的时间间隔)在环线上不断运行。在 0 时刻时,运输车在出发站点。

    \item 运输车上配备了容量足够大的制冷系统,疫苗可以在车上长时间存放。但是{\heiti{换乘}}(从一条线路切换到另一条线路)必须在同一个站点同一个时刻发生——因为各个站点没有独立的制冷系统,疫苗不能在站点内下车等待。

\end{enumerate}

现在 X 市想要从 1 号站点开始,经过若干条线路的运输和换乘,将疫苗运输到各个其他站点。
与其他站点不同,1 号站点配有冷库。也就是说,从 0 时刻开始,可以在 1 号站点等待某条线路运输车的到来,再开始疫苗运输。
问对于 2 号 ~ $n$ 号站点,分别最早可以在什么{\heiti{时刻}}将疫苗送到该站点。

注意:每个问题是独立的,即只需要求出 1 号站点到各个站点的最早送达时刻。


\subsection*{输入格式}

第一行两个整数 $n$, $m$。

接下来 $m$ 行,每行表示一条物流线路。对于第 $i ~ (1 \le i \le m)$ 条线路,首先有一个整数 $l_i ~ (2 \le l_i \le n)$ 表示该线路经过的站点个数。接下来 $2 l_i$ 个整数,第 $2j-1$ 和第 $2j$ 个整数分别表示该线路的第 $j ~ (1 \le j \le l_i)$ 个站点的编号 $a_{i,j} ~ (1 \le a_{i,j} \le n)$,以及该线路的第 $j$ 个站点到下一个站点所需的时间 $t_{i,j} ~ (1 \le t_{i,j} \le T)$(对于第 $l_i$ 个站点即为它到第 1 个站点的时间)。其中,每条线路的第 1 个站点为其出发站点。输入中同一行相邻的整数,均用一个空格隔开。


\subsection*{输出格式}

输出 $n - 1$ 行,第 $i$ 行表示将疫苗送达第 $i+1$ 个站点的最早时间:
如果能在有限时间内送达,输出最早的送达时刻;否则输出 \verb|inf|。

\examplebox{\lstinputlisting[frame=none]{data/22/5-1.in}}{\lstinputlisting[frame=none]{data/22/5-1.out}}

\examplebox*{\lstinputlisting[frame=none]{data/22/5-2.in}}{\lstinputlisting[frame=none]{data/22/5-2.out}}

在此样例中,有 5 个站点、3 条线路。第一条线路经过站点 1、2、3,第二条线路经过站点 3、4、5,第三条线路经过站点 3 和 5。

以下为从 1 号站点到各个其他站点的最早送达路线:

\begin{itemize}
    \item 2 号站点:通过第一条线路运输,在 100 时刻到达 2 号站点;
    \item 3 号站点:通过第一条线路运输,在 200 时刻到达 3 号站点;
    \item 4 号站点:通过第一条线路运输,在 500 时刻到达 3 号站点,然后换乘第三条线路,在 1500 时刻再次到达 3 号站点,最后换乘第二条线路,在 1600 时刻到达 4 号站点;
    \item 5 号站点:通过第一条线路运输,在 500 时刻到达 3 号站点,然后换乘第三条线路,在 625 时刻到达 5 号站点。
\end{itemize}

\examplebox{\lstinputlisting[frame=none]{data/22/5-3.in}}{\lstinputlisting[frame=none]{data/22/5-3.out}}

\subsection*{子任务}

对于 10\% 的数据,$n\le 5,m=1,T\le 10$。

对于 30\% 的数据,$n\le 5,m\le 2,T\le 10$。

对于 50\% 的数据,$n\le 5,m\le 5,T\le 10$。

对于 70\% 的数据,$n\le 10,m\le 10,T\le 100$。

对于 80\% 的数据,$n\le 30,m\le 30,T\le 1000$。

对于 95\% 的数据,$n\le 100,m\le 100,T\le 10^5$。

对于 100\% 的数据,$n\le 500,m\le 500,T\le 10^6$。