\subsection*{题目描述}

一幅长宽分别为 $n$ 个像素和 $m$ 个像素的灰度图像可以表示为一个 $n \times m$ 大小的矩阵 $A$。

其中每个元素 $A_{ij}$($0 \le i <n$、$0 \le j < m$)是一个 $[0, L)$ 范围内的整数,表示对应位置像素的灰度值。

具体来说,一个 $8$ 比特的灰度图像中每个像素的灰度范围是 $[0, 128)$。

一副灰度图像的灰度统计直方图(以下简称“直方图”)可以表示为一个长度为 $L$ 的数组 $h$,其中 $h[x]$($0 \le x < L$)表示该图像中灰度值为 $x$ 的像素个数。显然,$h[0]$ 到 $h[L-1]$ 的总和应等于图像中的像素总数 $n \cdot m$。

已知一副图像的灰度矩阵 $A$,试计算其灰度直方图 $h[0], h[1], \cdots, h[L-1]$。


\subsection*{输入格式}

输入共 $n + 1$ 行。

输入的第一行包含三个用空格分隔的正整数 $n$、$m$ 和 $L$,含义如前文所述。

第二到第 $n + 1$ 行输入矩阵 $A$。

第 $i + 2$($0 \le i < n$)行包含用空格分隔的 $m$ 个整数,依次为 $A_{i0}, A_{i1}, \cdots, A_{i(m-1)}$。


\subsection*{输出格式}

输出仅一行,包含用空格分隔的 $L$ 个整数 $h[0], h[1], \cdots, h[L-1]$,表示输入图像的灰度直方图。


\subsection*{样例}

输入\#1:

\begin{lstlisting}
4 4 16
0 1 2 3
4 5 6 7
8 9 10 11
12 13 14 15
\end{lstlisting}

输出\#1:

\begin{lstlisting}
1 1 1 1 1 1 1 1 1 1 1 1 1 1 1 1
\end{lstlisting}

输入\#2:

\begin{lstlisting}
7 11 8
0 7 0 0 0 7 0 0 7 7 0
7 0 7 0 7 0 7 0 7 0 7
7 0 0 0 7 0 0 0 7 0 7
7 0 0 0 0 7 0 0 7 7 0
7 0 0 0 0 0 7 0 7 0 0
7 0 7 0 7 0 7 0 7 0 0
0 7 0 0 0 7 0 0 7 0 0
\end{lstlisting}

输出\#2:

\begin{lstlisting}
48 0 0 0 0 0 0 29
\end{lstlisting}

\subsection*{子任务}

全部的测试数据满足 $0 < n, m \le 500$ 且 $4 \le L \le 256$。


